\documentclass[conference]{IEEEtran}

\ifCLASSINFOpdf
   \usepackage[pdftex]{graphicx}
  % declare the path(s) where your graphic files are
   \graphicspath{{../pdf/}{../jpeg/}}
  % and their extensions so you won't have to specify these with
  % every instance of \includegraphics
   \DeclareGraphicsExtensions{.pdf,.jpeg,.png}
\else
\fi

% correct bad hyphenation here
\hyphenation{op-tical net-works semi-conduc-tor}

% CUSTOM PACKAGES
%\usepackage{lettrine}
\usepackage{amsmath}
\renewcommand{\arraystretch}{1.3}
\usepackage{etoolbox}
\apptocmd{\thebibliography}{\setlength{\itemsep}{0.6pt}}{}{}
\usepackage{url}
\usepackage{caption}
\usepackage{subcaption}
\usepackage{colortbl}
\usepackage{xcolor}
\definecolor{light-gray}{gray}{0.78}
\usepackage{epstopdf}
\usepackage{lettrine}


\begin{document}

\title{A Tool-chain for Instruction Set Architecture Design}

\author{
\IEEEauthorblockN{Alessandro de Gennaro}
\IEEEauthorblockA{School of Electrical and\\Electronic Engineering\\
Newcastle University\\
Newcastle upon Tyne, United Kingdom\\
Email: a.de-gennaro@ncl.ac.uk}
\and
\IEEEauthorblockN{Paulius Stankaitis}
\IEEEauthorblockA{School of Computing Science\\
Newcastle University\\
Newcastle upon Tyne, United Kingdom\\
Email: paulius.stankaitis@ncl.ac.uk}}

\maketitle

\begin{abstract}
Adopting a systematic approach for the design of a processor instruction set is
instrumental for tackling the complexity, which rules most of modern processors.

We present a tool-chain which follows the designer through all the phases of the
design, from the specification of instructions to the hardware synthesis of a
microcontroller. The flow is also meant to simplify the understanding of the
Instruction Set Architecture (ISA) reference manuals. Often, these documents are
semi-formal, hard to read and fully understand. We believe that designers will
benefit from a visual graph-based model, automatically derived from the ISA
specification, and customisable to fit different needs. Some of the tools have
already been developed and tested on ARMv6-M architecture, others yet need to be
fully implemented.

The design flow will be integrated inside the Workcraft framework. We also compare
the presented approach with the others available in the literature.
\end{abstract}

\IEEEpeerreviewmaketitle


\section{Introduction}
\label{sec:intro}
Technology progress allows industries to integrate always more transistors over the
same amount of area, following Moore's law. In turn, design complexity of these
systems progressively increases. This led research to be focused on raising the
level of abstraction of the languages used at the early-stages of design. This work
presents a design flow for an easier specification, visualisation, simulation,
customisation and hardware synthesis of instruction sets.

The consultation of an ISA reference manual (i.e. ARMv6-M \cite{armManual}) can be
a difficult and tedious operation. Anthony Fox, in his attempt to describe a model
of the ARMv7 instruction set, argues: ``official reference manuals are large,
stretching to many hundreds of pages - one can easily overlook subtle details or
become bogged down with ``uninteresting'' background information'' (\cite{armv7},
section 1). And yet: ``official descriptions are semi-formal (ambiguous)''
(\cite{armv7}, section 1). 

In the light of the above, a simple and formal way to specify and, more
specifically, visualise instruction sets is needed. A visual graph-based model can
help designers for a quicker comprehension of the processor. ISAs in fact provide a
software level description of the hardware itself. We use Conditional Partial Order
Graph (CPOG) \cite{cpog},\cite{andreyPhd} as the visualisation model. They can
efficiently represent concurrent and sequential behaviours, and already come with a
tested tool-kit for the customisation, encoding and hardware synthesis
(\cite{workcraft}, \cite{satEncoding}, \cite{acsd}).

We propose a new domain-specific language for ISA specification. The current
implementation is embedded in \textit{Haskell} This is a recent and extremely
flexible functional language, and provides predefined constructs and classes for
our purpose (i.e. monad class).

The article is organised as follows: Section \ref{sec:flow} describes such a design
flow, the interactions between the various steps and how this can be helpful. In
Section \ref{sec:functionalities}, all the requirements which are and will be
supported by the tool chain are discussed. Section \ref{sec:arm} presents a case-
study: a subset of ARMv6-M ISA. And finally, Section \ref{sec:conclusion}
summarises the achieved results, presenting two more case and outlining the future
research directions.

\subsection{Related work}

An attempt to use partial orders to describe the instructions of a complex hardware
structure can be found in \cite{maxPhd}. The author uses Conditional Partial Order
Graphs to visually describe the instructions of the Intel 8051 Microcontroller.
Yet, he used them for building an asynchronous controller and managing the internal
execution flow of the datapath elements. The manual construction of partial orders
is an error-prone process. Connecting all the operations taking into account their
dependencies and order is a complex task. This further inspired our research, and
led us to bridge this gap with an automated approach.

Regarding the language we chose to adopt, it is not difficult to find other cases
where functional languages have been used as a starting point for ISA
specification. In this direction, \cite{isaFunc} provides an interesting attempt to
build an infrastructure for instruction set development. The authors developed the
concepts of \textit{state} and \textit{transformations}. The former represents the
current state of the machine, which evolves over time according to the
transformations (instructions). \textit{F. Yuan and K. I. Eder} instead, created a
formal and hierarchical model, which can be refined for fitting the needs of a
particular ISA. This model (characterised in \cite{isaEventB}) is composed by 4
abstract layers. Each of these layers describes a particular aspect of the
instruction set. The deeper the designer explores this hierarchical representation,
the more refined will be the final system.

Another example can be found in \cite{armv7}. The authors here build a framework
which can be extended to tailor instruction sets. They use a monadic programming
style, based on three basic operations: \textit{return}, \textit{bind} and
\textit{parallel}. These are meant to mimic the flow of an entity (the hardware
system), which is always returned as a result of the execution of an instruction.
Instructions can be bound together either sequentially or in parallel. Here, the
case study used is the ARMv7, a widespread instruction set embedded by the 
Cortex-A8 processor. An additional example of how an ISA might be specified inside
a tool is present in \cite{isaXml}. Both the modules and the instructions here are
introduced via a xml-based language, fairly readable but not very flexible.

%------------------------------------------------

\section{Design flow for ISA}
\label{sec:flow}
\section{Supported Functionalities}
\label{sec:functionalities}
\section{Case Study (ARMv6-M)}
\label{sec:arm}
\section{Conclusion}
\label{sec:conclusion}

\noindent\textbf{Acknowledgements:} The authors would like to thank Dr. Andrey Mokhov for helping with this work.

\begin{thebibliography}{1}

\bibitem{cpog}
	A. Mokhov, A. Yakovlev. \emph{``Conditional partial order graphs: Model,
	synthesis, and application''}. IEEE Transactions on Computers, Volume 59,
	Pages 1480-1493, November 2010.
	
\bibitem{andreyPhd}
	A. Mokhov. \emph{``Conditional Partial Order Graphs''}. Ph.D. Thesis,
	Newcastle University, September 2009.	

\bibitem{workcraft}
	I. Poliakov, D. Sokolov, A. Mokhov. \emph{``Workcraft: A static data flow
	structure editing, visualisation and analysis tool''}. Petri Nets and Other
	Models of Concurrency - ICATPN 2007. Pages 505-514, 2007.
	
\bibitem{satEncoding}
	A Mokhov, A Alekseyev, A Yakovlev. \emph{``Encoding of processor instruction
	sets with explicit concurrency control''}. Computers \& Digital Techniques,
	IET. Volume 5, Pages 427-439. November 2011.
	
\bibitem{acsd}
	A. de Gennaro, P. Stankaitis, A. Mokhov. \emph{``A Heuristic Algorithm for
	Deriving Compact Models of Processor Instruction Sets''}. 15th International
	Conference on Application of Concurrency to System Design (In Press). June
	24-26, 2015.

\bibitem{armv7}
	A. Fox, M. Myreen. \emph{``A Trustworthy Monadic Formalization of the ARMv7
	Instruction Set Architecture''}. Interactive Theorem Proving (ITP), pages
	243-258, 2010.	
	
\bibitem{isaEventB}
	F. Yuan, K. Eder. \emph{``A Generic Instruction Set Architecture Model in
	Event-B for Early Design Space Exploration''}. Technical Report CSTR-09-006,
	University of Bristol, September 2009.

\bibitem{isaFunc}
	T. A.Cook, E. Harcourt. \emph{``A Functional Specification Language for
	Instruction Set Architectures''}. Proceedings of the 1994 International
	Conference on Computer Languages. Publisher IEEE, pages 11-19. May 1994.
	
\bibitem{isaXml}
	A. Abbas, A. Ahmed, A. Ahmed, W. Uz Zaman Bajwa, A. Anwar, S. Abbasi. 
	emph{``A retargetable tool-suite for the design of application specific
	instruction set processors using a machine description language''}. IEEE
	International Symposium on Circuits and Systems, 2002. Volume 1, pages 425-428.
	ISCAS 2002.
	
\bibitem{maxPhd}
	M. Rykunov. \emph{``Design of Asynchronous Microprocessor for Power
	Proportionality''}. Ph.D. Thesis, Newcastle University. Technical Report Series
	NCL-EEE-MICRO-TR-2013-182, December 2013.
	
\bibitem{armManual}
	ARM Ltd. \emph{``ARMv6-M Architecture Reference Manual''}. 
	ARM DDI 0419C (ID092410), 2010.
	
\end{thebibliography}

\end{document}